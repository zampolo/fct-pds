\documentclass[
size=17pt,
paper=smartboard,
mode=present,
display=slidesnotes,
style=paintings,
nopagebreaks,
blackslide,
fleqn]{powerdot}

% styles: sailor, paintings
% wj capsules prettybox
% mode = handout or present


\newcommand{\palette}{Moitessier}
% palettes:
%    - sailor: Sea, River, Wine, Chocolate, Cocktail 
%    - paintings: Syndics, Skater, GoldenGate, Moitessier, PearlEarring, Lamentation, HolyWood, Europa, MayThird, Charon 

\newcommand{\cursopequeno}{EC01045 PDS}
\newcommand{\cursogrande}{\Large EC01045 -- Processamento digital de sinais}

\usepackage{amsmath,graphicx,color,amsfonts}
\usepackage[brazilian]{babel}
\usepackage[utf8]{inputenc}
\usepackage{bbding}

\author{Ronaldo de Freitas Zampolo\\FCT-ITEC-UFPA}
\date{ERE 2020}


\pdsetup{
   lf = {\cursopequeno},
   rf = {Apresentação do curso}, palette = {\palette}, randomdots={false}
}

%opening
\title{\cursogrande\\ \vspace{1cm}Apresentação do curso}

\begin{document}
   \maketitle[randomdots={false}]
   \begin{slide}{Agenda}
      \tableofcontents[content=sections]
   \end{slide}

   \section[ slide = false ]{Professor }
      \begin{slide}[toc=]{Professor}
         \begin{itemize}
            \item Professor: Ronaldo de Freitas Zampolo
            \item Afiliação:\\
                  Laboratório de Processamento de Sinais - LaPS\\
                  Faculdade de Engenharia da Computação e Telecomunicações - FCT\\
                  Instituto de Tecnologia - ITEC\\
                  Universidade Federal do Pará - UFPA
            \item Atendimento:\\
                  Sexta-feira: 11h00 - 12h00\\
                  Sala~32, altos do anexo do LEEC\\
                  \texttt{zampolo@ufpa.br}\\ 
                  %\texttt{zampolo@ieee.org}\\
                  \texttt{www.laps.ufpa.br/zampolo}
         \end{itemize}
      \end{slide}
      
   \section[ slide = false ]{Características do Curso}
      \begin{slide}[toc=]{Características do Curso}
         \begin{itemize}
            \item Carga horária: 60 h
            \item Aulas: terças e quintas, das 11h10 às 12h50
            \item Tópicos:
            \begin{itemize}
               \item Sinais e sistemas discretos (Cap. 2)
               \item Transformada Z (Cap. 3)
               \item Amostragem de sinais contínuos (Cap. 4)
               \item Análise de sistemas lineares invariantes no domínio transformado (Cap. 5)
               \item Estruturas para implementação de sistemas discretos (Cap. 6)
               \item Técnicas para projetos de filtros (Cap. 7)
               \item Transformada de Fourier discreta (Cap. 8)
            \end{itemize}
         \end{itemize}         
      \end{slide}
      
   \section[slide=false]{Objetivos do curso}
      \begin{slide}[toc=]{Objetivos do curso}
         \begin{itemize}
            \item Obter familiaridade:
            \begin{itemize}
               \item Matemática
               \item Conceitos
            \end{itemize}
            \item Utilizar ferramentas básicas
            \begin{itemize}
               \item Projeto
               \item Simulação
               \item Implementação
            \end{itemize}
            
         \end{itemize}
      \end{slide}

   \section[slide=false]{Aplicações}
      \begin{slide}[toc=]{Algumas situações de aplicação}
         \begin{itemize}
            \item Telecomunicações: equalizadores de canal, repetidores digitais, moduladores/demoduladores...
            \item Processamento de imagem/vídeo: restauração, codificação, avaliação de qualidade, segmentação, realce...
            \item Processamento de fala: identificação de locutor, reconhecimento, conversores texto-fala e fala-texto...
            \item Outros tipos de sinais: sísmico, temperatura, pressão...
         \end{itemize}
      \end{slide}
  
   \section[ slide = false ]{Bibliografia}
      \begin{slide}[toc=]{Bibliografia}
         \begin{itemize}
            \item Bibliografia básica
            \begin{itemize}
               \item \textbf{A. V. Oppenheim, R. W. Schafer. \emph{Processamento 
em tempo discreto de sinais}, 3ª Ed., Pearson, 2013.}
            \end{itemize}
            \item Bibliografia complementar
            \begin{itemize}
               \item P. S. R. Diniz, E. A. B. Silva, S. L. Netto. \emph{Processamento 
               Digital de Sinais - Projeto e análise de sistemas}, 2ª edição, Bookman Company, 2014.
               \item A. V. Oppenheim, A. S. Willsky. \emph{Sinais e Sistemas}, 2ª edição, Pearson, 2010.
            \end{itemize}
         \end{itemize}
      \end{slide}

   \section[ slide = false ]{Avaliação}
      \begin{slide}[toc=]{Instrumentos de avaliação e mapeamento nota-conceito}
         \begin{itemize}
            \item Avaliação continuada
            \item Trabalhos (Tr): em equipe ou individual com apresentação à turma
            \item Testes escritos (Te): em número de 3; 2ª chamada mediante requerimento
            \item Cálculo da nota ($N$):
            \begin{equation*}
               N=\frac{( \text{Tr} \times 4 + \text{Te} \times 6 )} {10}
            \end{equation*}
          \item Tabela de mapeamento:
            \begin{table}
               \centering
               \begin{tabular}{c|c}
                  \hline\hline
                  \textbf{Faixa} & \textbf{Conceito}\\
                  \hline
                  0 -- 4,9 & INS\\
                  5,0 -- 6,9 & REG\\
                  7,0 -- 8,9 & BOM\\
                  9,0 -- 10,0 & EXC\\
                  \hline\hline
               \end{tabular}
            \end{table}\end{itemize}
      \end{slide}
      
      \begin{slide}[toc=]{Datas dos testes escritos}
         \begin{itemize}
            \item Datas prováveis dos testes escritos:
            \begin{table}
               \centering
               \begin{tabular}{|l l|}
                  \hline
                  Teste 01: & 16/abril\\
                  Teste 02: & 21/maio\\
                  Teste 03: & 18/junho\\
                  %Teste 04: & 13/dezembro\\
                  %Substitutiva: & a definir\\
                  \hline
               \end{tabular}
            \end{table}
         \end{itemize}
      \end{slide}
\end{document}
