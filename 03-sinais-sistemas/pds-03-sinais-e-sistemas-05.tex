\documentclass[
size=17pt,
paper=smartboard,
mode=present,
display=slidesnotes,
style=sailor,
nopagebreaks,
blackslide,
fleqn]{powerdot}
 %wj capsules prettybox
 %mode = handout or present

\usepackage{amsmath,graphicx,color,amsfonts}
\usepackage[brazilian]{babel}
\usepackage[utf8]{inputenc}

\pdsetup{
   lf = {EC01045 PDS},
   rf = {Sinais e Sistemas Discretos}, palette = {Sea}, randomdots={false}
}


%opening
\title{\Large EC01045 -- Processamento Digital de Sinais\\ \vspace{1cm}Sinais e Sistemas Discretos}
\author{Ronaldo de Freitas Zampolo\\FCT-ITEC-UFPA}
\date{ }

\begin{document}
   \maketitle[randomdots={false}]
   \begin{slide}[toc=]{Agenda}
      \tableofcontents[content=sections]
   \end{slide}
   
\section[slide=false]{Propriedades da Transf. de Fourier}
\begin{slide}[toc=]{Sinais complexos}
\begin{itemize}
   \item Sequências  simétrica e anti-simétrica conjugadas
   \begin{align*}
      x[n] &= x_e[n]+x_o[n]\\
      x_e[n] &= \frac{1}{2} \left \{ x[n] + x^*[-n] \right \} = x_e^*[-n]\\
      x_o[n] &= \frac{1}{2} \left \{ x[n] - x^*[-n] \right \} = -x_o^*[-n]
   \end{align*} 
\end{itemize}
\end{slide}

\begin{slide}[toc=]{Sinais complexos 2}
\begin{itemize}
   \item Sequências  simétrica e anti-simétrica conjugadas
\begin{itemize}
   \item $x^*[n]\stackrel{F}{\leftrightarrow} X^*(e^{-j\omega})$
   \item $x^*[-n]\stackrel{F}{\leftrightarrow} X^*(e^{j\omega})$
   \item $\Re\{x[n]\} \stackrel{F}{\leftrightarrow} X_e(e^{j\omega})$
   \item $j\Im\{x[n]\} \stackrel{F}{\leftrightarrow} X_o(e^{j\omega})$
   \item $x_e[n] \stackrel{F}{\leftrightarrow} X_R(e^{j\omega}) = \Re\{X(e^{j\omega})\}$
   \item $x_o[n] \stackrel{F}{\leftrightarrow} jX_I(e^{j\omega}) = j\Im\{X(e^{j\omega})\}$
\end{itemize}
\end{itemize}
\end{slide}


\begin{slide}[toc=]{Sinais reais}
\begin{itemize}
   \item Para $x[n]$ real
   \begin{itemize}
      \item $ X(e^{j\omega}) = X^*(e^{-j\omega})$
      \item $ X_R(e^{j\omega}) = X_R(e^{-j\omega})$
      \item $ X_I(e^{j\omega}) = -X_I(e^{-j\omega})$
      \item $ |X(e^{j\omega})| = |X(e^{-j\omega})|$
      \item $ \angle X(e^{j\omega}) = -\angle X(e^{-j\omega})$
      \item $ x_o[n] \stackrel{F}{\leftrightarrow} X_R(e^{j\omega})$
      \item $ x_e[n] \stackrel{F}{\leftrightarrow} jX_I(e^{j\omega})$
   \end{itemize}
\end{itemize}
\end{slide}

\section[slide=false]{Teoremas da TF}
\begin{slide}[toc=]{Teoremas}
\center{
\small{\begin{tabular}{l|l}
\hline
Sequência & Transformada de Fourier \\
\hline
$x[n]$     &   $X(e^{j\omega})$ \\
$y[n]$     &   $Y(e^{j\omega})$\\
\hline
1. $ax[n]+by[n]$ & $aX(e^{j\omega})+bY(e^{j\omega})$\\
2. $x[n-n_d]$ & $e^{-j\omega n_d}X(e^{j\omega})$ \\
3. $e^{j\omega_o n}x[n]$ & $X(e^{j(\omega-\omega_o)})$\\
4. $x[-n]$ & $X(e^{-j\omega})$\\
           & $X^*(e^{j\omega})$ se $x[n]$ for real\\
5. $nx[n]$ & $j\frac{dX(e^{j\omega})}{d\omega}$\\
6. $x[n]*y[n]$ & $X(e^{j\omega})Y(e^{j\omega})$\\
7. $x[n]y[n]$ & $\frac{1}{2\pi}\int_{-\pi}^{\pi} X(e^{j\theta})Y(e^{j(\omega-\theta)})d\theta$\\
\hline
\multicolumn{2}{l}{8. $\sum_{n=-\infty}^{\infty}|x[n]|^2=\frac{1}{2\pi}\int_{-\pi}^{\pi}|X(e^{j\omega})|^2d\omega$}\\
\multicolumn{2}{l}{9. $\sum_{n=-\infty}^{\infty}x[n]y^*[n]=\frac{1}{2\pi}\int_{-\pi}^{\pi}X(e^{j\omega})Y^*(e^{j\omega})d\omega$}\\
\hline
\end{tabular}}}
\end{slide}

\begin{note}{Exercícios}
\begin{enumerate}
   \item Deduza cada uma das propriedades da transformada de Fourier.
   \item Prove cada um dos teoremas da transformada de Fourier. 
\end{enumerate}

\end{note}


\section[slide=false]{Sinais discretos aleatórios}
\begin{slide}[toc=]{Sinais determinísticos}
\begin{itemize}
   \item Definidos unicamente
   \begin{itemize}
      \item Expressão matemática
      \item Tabela
      \item Algum tipo de regra
   \end{itemize}
   
\end{itemize}
\end{slide}

\begin{slide}[toc=]{Sinais aleatórios}
\begin{itemize}
   \item Processos aleat\'orios $\times$ Vari\'aveis aleat\'orias
   \item Processo estacion\'ario
   \begin{itemize}
      \item Sentido estrito
      \item Sentido amplo
      \begin{align*}
         m_x[n] &= E\{x[n]\}\\
                &= k\\
         \phi_{xx}[n,n+m] &= E\{x[n]x[n+m]\}\\
                          &= \phi_{xx}[m]
      \end{align*}
   \end{itemize}
\end{itemize}
\end{slide}

\begin{slide}[toc=]{Entrada estacionária (EE) -- média}
\begin{itemize}
   \item Entrada estacion\'aria $\Rightarrow$ Sa\'ida estacion\'aria
   \item M\'edia
   \begin{align*}
      m_y[n] &= E\{y[n]\}\\
             &= E\left \{\sum_{k=-\infty}^{\infty} h[n-k] x[k]\right \}\\
             &= \sum_{k=-\infty}^{\infty} h[n-k] E\{x[k]\}\\
             &= m_x H\left (e^{j0}\right )
   \end{align*}
\end{itemize}
\end{slide}

\begin{slide}[toc=]{EE -- autocorrelação (AC) 1}
   \scriptsize{
   \begin{align*}
      \phi_{yy}[n,n+m] &= E\{y[n]y[n+m]\}\\
             &= E\left\{\sum_{k=-\infty}^{\infty} h[k] x[n-k]\sum_{r=-\infty}^{\infty} h[r] x[n+m-r]\right\}\\
             &= \sum_{k=-\infty}^{\infty} \sum_{r=-\infty}^{\infty} h[k]h[r] E\{x[n-k]x[n+m-r]\}\\
             &= \sum_{k=-\infty}^{\infty} \sum_{r=-\infty}^{\infty} h[k]h[r] \phi_{xx}[m-r+k]
             %&= \sum_{k=-\infty}^{\infty} \sum_{l=-\infty}^{\infty} h[k]h[l+k] \phi_{xx}[m-l]\\
             %&= \sum_{l=-\infty}^{\infty} \phi_{xx}[m-l]\sum_{k=-\infty}^{\infty} h[k]h[l+k] \\
             %&= \sum_{l=-\infty}^{\infty} \phi_{xx}[m-l] c_{hh}[l] \\
             %&= \phi_{xx}[m]*c_{hh}[m]\\
    %c_{hh}[n]&= h[n]*h[-n]  \quad\text{autocorr. determin\'istica}
   \end{align*}}
\end{slide}

\begin{slide}[toc=]{EE -- AC 2}
   \scriptsize{
   \begin{align*}
      \phi_{yy}[n,n+m] &= %E\{y[n]y[n+m]\}\\
             %&= E\left\{\sum_{k=-\infty}^{\infty} h[k] x[n-k]\sum_{r=-\infty}^{\infty} h[r] x[n+m-r]\right\}\\
             %&= \sum_{k=-\infty}^{\infty} \sum_{r=-\infty}^{\infty} h[k]h[r] E\{x[n-k]x[n+m-r]\}\\
             \sum_{k=-\infty}^{\infty} \sum_{r=-\infty}^{\infty} h[k]h[r] \phi_{xx}[m-r+k]\\
             &= \sum_{k=-\infty}^{\infty} \sum_{l=-\infty}^{\infty} h[k]h[l+k] \phi_{xx}[m-l]\\
             &= \sum_{l=-\infty}^{\infty} \phi_{xx}[m-l]\sum_{k=-\infty}^{\infty} h[k]h[l+k] \\
             &= \sum_{l=-\infty}^{\infty} \phi_{xx}[m-l] c_{hh}[l] \\
             &= \phi_{xx}[m]*c_{hh}[m]    
   \end{align*}}
   \begin{equation*}
      c_{hh}[n]= h[n]*h[-n]  \quad\text{(autocorrelação determin\'istica)} 
   \end{equation*}
\end{slide}



\begin{slide}[toc=]{EE -- densidade espectral de potência}
\begin{align*}
   \Phi_{yy}(e^{j\omega}) &= \sum_{m=-\infty}^{\infty}  \phi_{yy}[m]e^{-j\omega m}\\
                          &= \sum_{m=-\infty}^{\infty}  \{\phi_{xx}[m]*c_{hh}[m]\} e^{-j\omega m}\\
                          &= \Phi_{xx}(e^{j\omega}) C_{hh}(e^{j\omega})
\end{align*}
\begin{equation*}
     \boxed{\Phi_{yy}(e^{j\omega}) = \Phi_{xx}(e^{j\omega}) |H(e^{j\omega})|^2} 
  \end{equation*}
\end{slide}

% \begin{slide}{Sinais discretos aleat\'orios}
% \begin{itemize}
%    \item Densidade Espectral de Pot\^encia
%    \begin{align*}
%     %  \Phi_{yy}(e^{j\omega}) &= \sum_{m=-\infty}^{\infty}  \phi_{yy}[m]e^{-j\omega m}\\
%     %                         &= \sum_{m=-\infty}^{\infty}  \{\phi_{xx}[m]*c_{hh}[m]\} e^{-j\omega m}\\
%     %                         &= \Phi_{xx}(e^{j\omega}) C_{hh}(e^{j\omega})
%     C_{hh}(e^{j\omega})       &= \sum_{m=-\infty}^{\infty} c[m] e^{-j\omega m} \\
%                               &= \sum_{m=-\infty}^{\infty} \sum_{k=-\infty}^{\infty}h[k]h[m+k] e^{-j\omega m} \\
%                               &= \sum_{k=-\infty}^{\infty} h[k]\sum_{m=-\infty}^{\infty}h[m+k] e^{-j\omega m} 
%                               %&= H(e^{j\omega})\sum_{k=-\infty}^{\infty} h[k]e^{j\omega k} \\
%                               %&= H(e^{j\omega})H^*(e^{j\omega})\\
%                               %&= |H(e^{j\omega})|^2
%    \end{align*}
% \end{itemize}
% \end{slide}
% 
% \begin{slide}{Sinais discretos aleat\'orios}
% \begin{itemize}
%    \item Densidade Espectral de Pot\^encia
%    \begin{align*}
%     %  \Phi_{yy}(e^{j\omega}) &= \sum_{m=-\infty}^{\infty}  \phi_{yy}[m]e^{-j\omega m}\\
%     %                         &= \sum_{m=-\infty}^{\infty}  \{\phi_{xx}[m]*c_{hh}[m]\} e^{-j\omega m}\\
%     %                         &= \Phi_{xx}(e^{j\omega}) C_{hh}(e^{j\omega})
%     C_{hh}(e^{j\omega})       %&= \sum_{m=-\infty}^{\infty} c[m] e^{-j\omega m} \\
%                               %&= \sum_{m=-\infty}^{\infty} \sum_{k=-\infty}^{\infty}h[k]h[m+k] e^{-j\omega m} \\
%                               %&= \sum_{k=-\infty}^{\infty} h[k]\sum_{m=-\infty}^{\infty}h[m+k] e^{-j\omega m} 
%                               &= H(e^{j\omega})\sum_{k=-\infty}^{\infty} h[k]e^{j\omega k} \\
%                               &= H(e^{j\omega})H^*(e^{j\omega})\\
%                               &= |H(e^{j\omega})|^2\\
%     \Phi_{yy}(e^{j\omega}) &= \Phi_{xx}(e^{j\omega}) C_{hh}(e^{j\omega})
%     %\Phi_{yy}(e^{j\omega}) &= \Phi_{xx}(e^{j\omega}) |H(e^{j\omega})|^2
%   \end{align*}
%      
%   \begin{equation*}
%      \boxed{\Phi_{yy}(e^{j\omega}) = \Phi_{xx}(e^{j\omega}) |H(e^{j\omega})|^2}
%   \end{equation*}
% \end{itemize}
% \end{slide}

\begin{slide}[toc=]{EE -- Potência}
\begin{itemize}
   \item C\'alculo da pot\^encia de um sinal
   \begin{align*}
      E\{y^2[n]\} &= E\{y[n]y[n]\}\\
                  &= \phi_{yy}[0]\\
                  &= \frac{1}{2\pi}\int_{-\pi}^{\pi}\Phi_{yy}(e^{j\omega})d\omega 
   \end{align*}
\end{itemize}
\end{slide}

\end{document}
