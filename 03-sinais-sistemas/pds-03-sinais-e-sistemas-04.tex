\documentclass[
size=17pt,
paper=smartboard,
mode=present,
display=slidesnotes,
style=sailor,
nopagebreaks,
blackslide,
fleqn]{powerdot}
 %wj capsules prettybox
 %mode = handout or present

\usepackage{amsmath,graphicx,color,amsfonts}
\usepackage[brazilian]{babel}
\usepackage[utf8]{inputenc}

\pdsetup{
   lf = {EC01045 PDS},
   rf = {Sinais e Sistemas Discretos}, palette = {Sea}, randomdots={false}
}


%opening
\title{\Large EC01045 -- Processamento Digital de Sinais\\ \vspace{1cm}Sinais e Sistemas Discretos}
\author{Ronaldo de Freitas Zampolo\\FCT-ITEC-UFPA}
\date{ }

\begin{document}
   \maketitle[randomdots={false}]
   \begin{slide}[toc=]{Agenda}
      \tableofcontents[content=sections]
   \end{slide}

\section[slide=true]{Equações de diferenças}
\begin{slide}[toc=]{Definições}
   \begin{itemize}
    \item Equação de diferenças de $N$-ésima ordem e coeficientes constantes:
    \begin{equation*}
       \sum_{ k = 0 }^{ N } a_k y[ n - k ]=\sum_{ m = 0 }^{ M } b_m x[ n - m ],
    \end{equation*}
    onde $x[n]$ e $y[n]$ são entrada e saída do sistema.
   \end{itemize}
\end{slide}

\begin{slide}[toc=]{Exemplos}
   \begin{itemize}
   \item Acumulador
    \begin{equation*}
       %\begin{split}
         y[n] = \sum_{ k = -\infty}^{n} x[k]= y[n-1] + x[n]
       %\end{split}
   \end{equation*}\pause
    \item Média móvel
    \begin{align*}
       %\begin{split}
         y[n] &= \frac{1}{M_2+1}\sum_{ k = 0}^{M_2} x[n-k]\\
         h[n] &= \frac{1}{M_2+1}\left ( u[n] - u[n-M_2-1] \right )\\
              &= \frac{1}{M_2+1}\left ( \delta[n] - \delta[n-M_2-1] \right )\ast u[n]
       %\end{split}
    \end{align*}
   \end{itemize}
\end{slide}


\begin{slide}[toc=]{Resolução}
   \begin{itemize}
      \item Pode-se rescrever a eq. de diferenças como
      \begin{equation*}
         \sum_{ k = 0 }^{ N } a_k y[ n - k ]-\sum_{ m = 0 }^{ M } b_m x[ n - m ]=0
      \end{equation*}\pause
      \item Há dois tipos de solução:
      \begin{itemize}
         \item Solução particular ($y_p[n]$): quando $x[n]\neq 0$
         \item Solução homogênea ($y_h[n]$): quando $x[n]= 0$
      \end{itemize}
   \end{itemize}
\end{slide}

\begin{slide}[toc=]{Solução particular e homogênea}
   \begin{itemize}
      \item Solução particular ($x[n]\neq 0$)
      \begin{equation*}
         \sum_{ k = 0 }^{ N } a_k y_p[ n - k ]-\sum_{ m = 0 }^{ M } b_m x[ n - m ]=0
      \end{equation*}\pause
      \item Solução homogênea ($x[n]=0$)
      \begin{equation*}
         \sum_{ k = 0 }^{ N } a_k y_h[ n - k ]=0
      \end{equation*}
   \end{itemize}
\end{slide}

\begin{slide}[toc=]{Solução completa}
   \begin{itemize}
      \item A soma das soluções particular e homogênea também é solução da equação de diferenças
      \begin{equation*}
         \sum_{ k = 0 }^{ N } a_k \{y_p[ n - k ]+y_h[ n - k ]\}-\sum_{ m = 0 }^{ M } b_m x[ n - m]=0
      \end{equation*}\pause
      \item Graus de liberdade ($N$): 
      \begin{itemize}
         \item Depende de $N$ constantes arbitrárias
         \item Necessita de $N$ condições iniciais auxiliares
      \end{itemize}
   \end{itemize}
\end{slide}

\begin{slide}[toc=]{Determinação da solução homogênea}
   \begin{itemize}
      \item A solução homogênea é da forma
      \begin{equation*}
         y_h[n]=\sum_{m=1}^N A_m z_m^n
      \end{equation*}
      \item Assim, os números complexos $z_m$ devem ser raízes do polinômio
      \begin{equation*}
         \sum_{k=0}^N a_k z^{-k}
      \end{equation*}
      \item Após a determinação dos $z_m$, as constantes $A_m$ devem ser calculadas com o auxílio 
      de $N$ condições auxiliares.
   \end{itemize}
\end{slide}

\begin{slide}[toc=]{Discussão}
   \begin{itemize}
      \item Para um sistema cuja entrada e saída satisfazem uma equação de 
      diferenças linear com coeficientes constantes:
      \begin{itemize}
         \item A saída para uma dada entrada não é única, depende de condições auxiliares
         \item A linearidade, a invariância e a causalidade dependerão das condições auxiliares 
         (sistema em repouso ou não)
      \end{itemize}
      \item Caso especial: $N=0$
      \begin{itemize}
         \item Não há necessidade de recursão, nem de condições iniciais
         \item A saída é dada pela expressão
         \begin{equation*}
            y[n]=\sum_{k=0}^M\left (\frac{b_k}{a_0}\right )x[n-k]
         \end{equation*}
      \end{itemize}
   \end{itemize}
\end{slide}

\begin{slide}[toc=]{Discussão}
   \begin{itemize}
      \item Caso especial: $N=0$
      \begin{itemize}
         \item Resposta ao impulso finita (FIR)
         \begin{equation*}
            y[n]=\begin{cases}
                    \frac{b_k}{a_0}, & 0\leq n \leq M,\\
                    0              , & \text{caso contrário}
                 \end{cases}
         \end{equation*}
      \end{itemize}
   \end{itemize}
\end{slide}


\section[slide=true]{Representação em frequência}
 \begin{slide}[toc=]{Auto-funções}
\begin{itemize}
 \item Funções exponenciais complexas como ``auto-funções'' de um sistema LI
 %\footnotesize{
 \begin{align*}
   x[n]&=e^{j\omega n}\\
   y[n]&=\sum_{k=-\infty}^{\infty}h[k]e^{j\omega (n-k)}\\
       &=e^{j\omega n}\left [\sum_{k=-\infty}^{\infty}h[k]e^{-j\omega k} \right ]\\
       &= H(e^{j\omega})e^{j\omega n}
 \end{align*}%}
 $H(e^{j\omega})$: \textcolor{red}{resposta em frequência do sistema.}
\end{itemize}
\end{slide}

\begin{slide}[toc=]{Exemplos}
\begin{itemize}
   \item Ache a $H(e^{j\omega})$ do sistema $y[n]=x[n-n_d]$.
   \item Ache a resposta do sistema anterior para \begin{equation*}
                                                     x[n]=\sum_k \alpha_k e^{j\omega_k n}.
                                                    \end{equation*}
   \item Calcule $y[n]$ em função de $H(e^{j\omega})$ de um sistema qualquer, dado $x[n]=A\cos(\omega_o n+\phi)$.
   \item Mostre que $H(e^{j\omega})$  periódico em $2\pi$.
\end{itemize}
\end{slide}

\section[slide=true]{Transformada de Fourier}
\begin{slide}[toc=]{Transformada de Fourier}
 \begin{itemize}
  \item Defini\c c\~oes
     \begin{equation*} X(e^{j\omega})=\sum_{n=-\infty}^{\infty} x[n]e^{-j\omega n}  \end{equation*}
     \begin{equation*} x[n]=\frac{1}{2\pi}\int_{-\pi}^{\pi} X(e^{j\omega})e^{j\omega n} d\omega \end{equation*}
   \item Forma retangular: $X(e^{j\omega}) = X_R(e^{j\omega}) + j X_I(e^{j\omega})$
   \item Forma polar: $X(e^{j\omega}) = |X(e^{j\omega})|e^{j\angle X(e^{j\omega})}$
 \end{itemize}
\end{slide}

\begin{slide}[toc=]{Prova da transformada inversa}
%  \begin{itemize}
%   \item Prova da transformada inversa
      \begin{align*} x[n]&= \frac{1}{2\pi}\int_{-\pi}^{\pi} X(e^{j\omega})e^{j\omega n} d\omega\\
                       &= \frac{1}{2\pi}\int_{-\pi}^{\pi} \left ( \sum_{m=-\infty}^{\infty} x[m]e^{-j\omega m}\right )e^{j\omega n} d\omega\\
                       & = \sum_{m=-\infty}^{\infty}x[m]\left ( \frac{1}{2\pi} \int_{-\pi}^{\pi} e^{j\omega(n-m)}d\omega \right ) \\   
                       & = \sum_{m=-\infty}^{\infty}x[m] \delta[n-m] = x[n]\end{align*}
%  \end{itemize}
\end{slide}

\begin{slide}[toc=]{Exemplos 2}
Calcule a resposta em frequência dos sistemas LI abaixo representados:
\begin{enumerate}
   \item $h_1[n] = a^nu[n], \quad |a|<1$
   \item $h_2[n] = a^{|n|}, \quad |a|<1$
   \item $h_3[n] = \begin{cases} 1,& |n|\leq N_1\\0,&|n|> N_1\end{cases}$
\end{enumerate}
\end{slide}

\begin{note}[toc=]{Exercícios}
Trace os gráficos de magnitude e fase das respostas em frequência dos sistemas de resposta ao impulso $h_1$, $h_2$ e $h_3$ para: 
\begin{enumerate}
   \item $a = 0,9$ ($h_1$ e $h_2$)
   \item $a = -0,9$ ($h_1$ e $h_2$)
   \item $N_1 = 3$ ($h_3$)
   \item $N_1 = 10$  ($h_3$)
\end{enumerate}
Qual o comportamento esperado da transformada de Fourier de $h_3[n]$ quando $N_1\rightarrow 0$ ?
\end{note}

\end{document}
